\newcommand{\imagem}[1]{
    \centering
    \begin{subfigure}{0.32\textwidth}
        \centering
        \includegraphics[width=4.4cm]{imagens/#1.png}
        \caption{\texttt{figuras/#1.png}}
        \label{fig:#1:orig}
    \end{subfigure}%
    \begin{subfigure}{0.32\textwidth}
        \centering
        \includegraphics[width=4.4cm]{resultados/var/unidir_#1.png}
        \caption{Varredura unidirecional.}
        \label{fig:#1:unidir}
    \end{subfigure}%
    \begin{subfigure}{0.32\textwidth}
        \centering
        \includegraphics[width=4.4cm]{resultados/var/alternada_#1.png}
        \caption{Varredura alternada.}
        \label{fig:#1:floyd}
    \end{subfigure}
}

\begin{figure}[H]
    \centering
    \imagem{baboon}\\[8pt]
    \imagem{peppers}\\[8pt]
    \imagem{monalisa}\\[8pt]
    \imagem{watch}

    \caption{Aplicação da distribuição de Floyd e Steinberg com as duas varreduras.}
    \label{fig:resultado:varredura}
\end{figure}

Na \cref{fig:resultado:varredura}, os resultados das duas varreduras são visualmente muito semelhantes e asmétricas da \cref{tab:varedura} corroboram com essa ideia, mostrando uma variação pequena de uma opção para a outra. A maior diferença visual aparece na imagem \texttt{peppers.png}, onde a varredura unidirecional (\ref{fig:peppers:unidir}) gera pequenos artefatos em formato de linhas horizontais, principalmente no pimentão vermelho.

\begin{table}[H]
    \centering
    \caption{Comparativo entre os resultados da \cref{fig:resultado:varredura}.}
    \label{tab:varedura}

    \begin{tabular}{cc|cccc}
        \toprule
        Figura & Varredura & RMSE & SNR (dB) & PSNR (dB) & Correlação \\
        \midrule
        \multirow{2}{*}{\texttt{baboon.png}} & Unidirecional & 10.367 & -0.004 & 27.818 & 0.480 \\
        & Alternada & 10.364 & -0.001 & 27.820 & 0.477 \\
        \midrule
        \multirow{2}{*}{\texttt{peppers.png}} & Unidirecional & 10.322 & -0.017 & 27.856 & 0.531 \\
        & Alternada & 10.325 & -0.020 & 27.853 & 0.531 \\
        \midrule
        \multirow{2}{*}{\texttt{monalisa.png}} & Unidirecional & 10.411 & 0.009 & 27.781 & 0.419 \\
        & Alternada & 10.407 & 0.013 & 27.784 & 0.418 \\
        \midrule
        \multirow{2}{*}{\texttt{watch.png}} & Unidirecional & 10.262 & 0.016 & 27.907 & 0.372 \\
        & Alternada & 10.263 & 0.015 & 27.905 & 0.371 \\
        \bottomrule
    \end{tabular}
\end{table}

Nos resultados seguintes, a imagens serão varridas de forma alternada, como este é o padrão da ferramenta.
