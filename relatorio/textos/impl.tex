\section{Implementação} \label{sec:impl}

\subsection{Técnica de Meios-Tons com Difusão de Erro}

    Essa técnica de pontilhado se baseia em alterar cada pixel com escala de 8 bits para escala de 2 bits, levando para o seu bit mais próximo. No entanto, ao longo do processo, o erro é calculado em relação ao novo pixel, que é então distribuído na sua vizinhaça, influenciando as aplicações nos pixels seguintes.

    Assim, quando um pixel resulta em valor muito distante do original, seus vizinhos são reduzidos ou incrementados, aumentando a chance de que eles sejam transformados para outro nível. Isso faz com que a vizinhaça mantenha um pouco mais da distribuição local de intensidade, fazendo com que imagem fique similar à original para o nosso sistema visual.

    A distribuição de erros pode ser feita de várias formas, como pode ser visto na \red{secao ??}.

%     \input{textos/impl/codigo}

% \subsection{Tratamento de Bordas}

%     Para que a convolução possa ser feita nas bordas da imagem, existem vários modos de tratamento. Na ferramenta foram implementadas três formas de dentre as várias possíveis: a extensão do último pixel (\ref{fig:borda:extensao}), a reflexão dos pixels de borda (\ref{fig:borda:reflexao}) e a mesma reflexão, mas sem repetir o pixel mais externo (\ref{fig:borda:reflexao-pulada}).

%     A \textit{flag} \mintinline{bash}{--borda}, ou \mintinline{bash}{-b}, serve para controlar o tratamento de borda. As opções devem ser passadas como aparecem na \cref{fig:borda}. Por padrão, o tratamento é feito como na \cref{fig:borda:reflexao-pulada}, como se fosse a opção \mintinline{bash}{-b reflexao_pula_ultimo}. Ambos \textit{backends}, SciPy e OpenCV, funcionam com os três tipos de bordas.

%     Apesar de interessante, e em alguns casos importante, o três tratamentos de bordas não alteram muito no resultado da convolução. A razão disso é que os \textit{kernels} desse trabalho são relativamente pequenos.

% \subsection{Discretização} \label{sec:impl:t}

%     Toda o processo de convolução é feito com operações de ponto flutuante, buscando evitar \textit{overflow} e problemas de arredondamento. Então, para que a matriz volte a representar uma imagem, é preciso discretizar os valores, para os níveis 0 a 255.

%     A forma mais comum é arrendondando os valores para os inteiros mais próximos no intervalo $[0, 255]$. Dessa forma, os valores negativos se tornam 0 e valores maiores que 255 vão para 255. No código, esse método foi implementado na função \pyline{lib.trunca}. Essa é a opção padrão do programa.

%     Uma forma alternativa também foi implementada, baseada no mapeamento linear do menor valor da imagem para 0 e do maior para 255. Na ferramenta em Python, esse método de discretização está implementado na função \pyline{lib.transforma_limites} e pode ser selecionada com a opção \mintinline{bash}{-t} na linha de comando. Essa opção não foi muito utilizada neste relatório.

% \subsection{Combinação das Imagens} \label{sec:impl:n}

%     Vários filtros podem ser passados como argumento, como no exemplo da \cref{sec:execucao}. Os filtros são aplicados, um por vez, na mesma imagem de entrada, discretizados e só então combinados em uma imagem final. A combinação é feita pela raiz da soma quadrática, em ponto flutuante, e discretizada novamente.

%     Para padronizar a implementação, a etapa de combinação é feita mesmo com apenas uma imagem. Por causa disso, a discretização é feita antes e depois da combinação, mantendo o resultado esperado da convolução. No entanto, isso pode ser alterado com a opção \mintinline{bash}{-n}, fazendo com que a discretização seja aplicada apenas no final.

%     Para um filtro apenas, isso faz com que a convolução seja tratada pelo valor absoluto, fazendo as imagens ficarem com regiões mais claras, onde anteriormente seria preto. Esse modo não foi utilizado neste relatório.
