\section{Implementação} \label{sec:impl}

\subsection{Técnica de Meios-Tons com Difusão de Erro}

    Essa técnica de pontilhado se baseia em alterar cada pixel com escala de 8 bits para escala de 2 bits, levando para o seu bit mais próximo. No entanto, ao longo do processo, o erro é calculado em relação ao novo pixel, que é então distribuído na sua vizinhaça, influenciando as aplicações nos pixels seguintes.

    Assim, quando um pixel resulta em valor muito distante do original, seus vizinhos são reduzidos ou incrementados, aumentando a chance de que eles sejam transformados para outro nível. Isso faz com que a vizinhaça mantenha um pouco mais da distribuição local de intensidade, fazendo com que imagem fique similar à original para o nosso sistema visual.

    A distribuição de erros pode ser feita de várias formas, como pode ser visto na \red{secao ??}.

\subsection{Formas de Varredura}

    \begin{figure}[H]
        \centering
        \newcommand{\imagem}[1]{
    \centering
    \begin{subfigure}{0.32\textwidth}
        \centering
        \includegraphics[width=4.4cm]{imagens/#1.png}
        \caption{\texttt{figuras/#1.png}}
        \label{fig:#1:orig}
    \end{subfigure}%
    \begin{subfigure}{0.32\textwidth}
        \centering
        \includegraphics[width=4.4cm]{resultados/var/unidir_#1.png}
        \caption{Varredura unidirecional.}
        \label{fig:#1:unidir}
    \end{subfigure}%
    \begin{subfigure}{0.32\textwidth}
        \centering
        \includegraphics[width=4.4cm]{resultados/var/alternada_#1.png}
        \caption{Varredura alternada.}
        \label{fig:#1:alt}
    \end{subfigure}
}

\begin{figure}[H]
    \centering
    \imagem{baboon}\\[8pt]
    \imagem{peppers}\\[8pt]
    \imagem{monalisa}\\[8pt]
    \imagem{watch}

    \caption{Aplicação da distribuição de Floyd e Steinberg com as duas varreduras.}
    \label{fig:resultado:varredura}
\end{figure}

Na \cref{fig:resultado:varredura}, os resultados das duas varreduras são visualmente muito semelhantes e asmétricas da \cref{tab:varedura} corroboram com essa ideia, mostrando uma variação pequena de uma opção para a outra. A maior diferença visual aparece na imagem \texttt{peppers.png}, onde a varredura unidirecional (\ref{fig:peppers:unidir}) gera pequenos artefatos em formato de linhas horizontais, principalmente no pimentão vermelho.

\begin{table}[H]
    \centering
    \caption{Comparativo entre os resultados da \cref{fig:resultado:varredura}.}
    \label{tab:varedura}

    \begin{tabular}{cc|cccc}
        \toprule
        Figura & Varredura & RMSE & SNR (dB) & PSNR (dB) & Correlação \\
        \midrule
        \multirow{2}{*}{\texttt{baboon.png}} & Unidirecional & 10.367 & -0.004 & 27.818 & 0.480 \\
        & Alternada & 10.364 & -0.001 & 27.820 & 0.477 \\
        \midrule
        \multirow{2}{*}{\texttt{peppers.png}} & Unidirecional & 10.322 & -0.017 & 27.856 & 0.531 \\
        & Alternada & 10.325 & -0.020 & 27.853 & 0.531 \\
        \midrule
        \multirow{2}{*}{\texttt{monalisa.png}} & Unidirecional & 10.411 & 0.009 & 27.781 & 0.419 \\
        & Alternada & 10.407 & 0.013 & 27.784 & 0.418 \\
        \midrule
        \multirow{2}{*}{\texttt{watch.png}} & Unidirecional & 10.262 & 0.016 & 27.907 & 0.372 \\
        & Alternada & 10.263 & 0.015 & 27.905 & 0.371 \\
        \bottomrule
    \end{tabular}
\end{table}

Nos resultados seguintes, a imagens serão varridas de forma alternada, como este é o padrão da ferramenta.


        \caption{Argumentos válidos para \mintinline{bash}{--varredura} ou \mintinline{bash}{-v}.}
        \label{fig:varredura}
    \end{figure}

    \begin{figure}
        \centering
        \begin{subfigure}{0.30\textwidth}
    \centering
    \begin{kmatrix}
    \matrix[square matrix]{
        ~ & $f(x, y)$ & 7/16 \\
        3/16 & 5/16 & 1/16 \\
    };
\end{kmatrix}
    \caption{Floyd e Steinberg}
\end{subfigure}%
\begin{subfigure}{0.63\textwidth}
    \centering
    \begin{kmatrix}
    \matrix[square matrix]{
        ~ & ~ & ~ & $f(x, y)$ & ~ & $\displaystyle\frac{32}{200}$ & ~ \\
        $\displaystyle\frac{12}{200}$ & ~ & $\displaystyle\frac{26}{200}$ & ~ & $\displaystyle\frac{30}{200}$ & ~ & $\displaystyle\frac{16}{200}$ \\
        ~ & $\displaystyle\frac{12}{200}$ & ~ & $\displaystyle\frac{26}{200}$ & ~ & $\displaystyle\frac{12}{200}$ & ~ \\
        $\displaystyle\frac{5}{200}$ & ~ & $\displaystyle\frac{12}{200}$ & ~ & $\displaystyle\frac{12}{200}$ & ~ & $\displaystyle\frac{5}{200}$ \\
    };
\end{kmatrix}
    \caption{Stevenson e Arce}
\end{subfigure}\\[8pt]
\begin{subfigure}{0.48\textwidth}
    \centering
    \begin{kmatrix}
    \matrix[square matrix]{
        ~ & ~ & $f(x, y)$ & 8/32 & 4/32 \\
        2/32 & 4/32 &  8/32 & 4/32 & 2/32 \\
    };
\end{kmatrix}
    \caption{Burkes}
\end{subfigure}%
\begin{subfigure}{0.48\textwidth}
    \centering
    \begin{kmatrix}
    \matrix[square matrix]{
        ~ & ~ & $f(x, y)$ & 5/32 & 3/32 \\
        2/32 & 4/32 &  5/32 & 4/32 & 2/32 \\
        ~ & 2/32 &  3/32 & 2/32 & ~ \\
    };
\end{kmatrix}
    \caption{Sierra}
\end{subfigure}\\[8pt]
\begin{subfigure}{0.48\textwidth}
    \centering
    \begin{kmatrix}
    \matrix[square matrix]{
        ~ & ~ & $f(x, y)$ & 8/42 & 4/42 \\
        2/42 & 4/42 &  8/42 & 4/42 & 2/42 \\
        1/42 & 2/42 &  4/42 & 2/42 & 1/42 \\
    };
\end{kmatrix}
    \caption{Stucki}
\end{subfigure}%
\begin{subfigure}{0.48\textwidth}
    \centering
    \begin{kmatrix}
    \matrix[square matrix]{
        ~ & ~ & $f(x, y)$ & 7/48 & 5/48 \\
        3/48 & 5/48 & 7/48 & 5/48 & 3/48 \\
        1/48 & 3/48 & 5/48 & 3/48 & 1/48 \\
    };
\end{kmatrix}
    \caption{Jarvis, Judice e Ninke}
\end{subfigure}
    \end{figure}

%     \input{textos/impl/codigo}

% \subsection{Tratamento de Bordas}

%     Para que a convolução possa ser feita nas bordas da imagem, existem vários modos de tratamento. Na ferramenta foram implementadas três formas de dentre as várias possíveis: a extensão do último pixel (\ref{fig:borda:extensao}), a reflexão dos pixels de borda (\ref{fig:borda:reflexao}) e a mesma reflexão, mas sem repetir o pixel mais externo (\ref{fig:borda:reflexao-pulada}).

%     A \textit{flag} \mintinline{bash}{--borda}, ou \mintinline{bash}{-b}, serve para controlar o tratamento de borda. As opções devem ser passadas como aparecem na \cref{fig:borda}. Por padrão, o tratamento é feito como na \cref{fig:borda:reflexao-pulada}, como se fosse a opção \mintinline{bash}{-b reflexao_pula_ultimo}. Ambos \textit{backends}, SciPy e OpenCV, funcionam com os três tipos de bordas.

%     Apesar de interessante, e em alguns casos importante, o três tratamentos de bordas não alteram muito no resultado da convolução. A razão disso é que os \textit{kernels} desse trabalho são relativamente pequenos.

% \subsection{Discretização} \label{sec:impl:t}

%     Toda o processo de convolução é feito com operações de ponto flutuante, buscando evitar \textit{overflow} e problemas de arredondamento. Então, para que a matriz volte a representar uma imagem, é preciso discretizar os valores, para os níveis 0 a 255.

%     A forma mais comum é arrendondando os valores para os inteiros mais próximos no intervalo $[0, 255]$. Dessa forma, os valores negativos se tornam 0 e valores maiores que 255 vão para 255. No código, esse método foi implementado na função \pyline{lib.trunca}. Essa é a opção padrão do programa.

%     Uma forma alternativa também foi implementada, baseada no mapeamento linear do menor valor da imagem para 0 e do maior para 255. Na ferramenta em Python, esse método de discretização está implementado na função \pyline{lib.transforma_limites} e pode ser selecionada com a opção \mintinline{bash}{-t} na linha de comando. Essa opção não foi muito utilizada neste relatório.

% \subsection{Combinação das Imagens} \label{sec:impl:n}

%     Vários filtros podem ser passados como argumento, como no exemplo da \cref{sec:execucao}. Os filtros são aplicados, um por vez, na mesma imagem de entrada, discretizados e só então combinados em uma imagem final. A combinação é feita pela raiz da soma quadrática, em ponto flutuante, e discretizada novamente.

%     Para padronizar a implementação, a etapa de combinação é feita mesmo com apenas uma imagem. Por causa disso, a discretização é feita antes e depois da combinação, mantendo o resultado esperado da convolução. No entanto, isso pode ser alterado com a opção \mintinline{bash}{-n}, fazendo com que a discretização seja aplicada apenas no final.

%     Para um filtro apenas, isso faz com que a convolução seja tratada pelo valor absoluto, fazendo as imagens ficarem com regiões mais claras, onde anteriormente seria preto. Esse modo não foi utilizado neste relatório.
